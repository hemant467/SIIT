Entities
======================================
.Entity is a special symbol
.every entity starts with '&' and ends with ';'.
. syntax:
==========
   &entityname;
   &#asciicode;
  
. Entity 
 ---------
  &lt;  <
  &gt;  >
  &amp; &
  &copy; copyright symbol
  &trade; trade mark symbol
  &reg; registered trade mark symbol
  &nbsp; no back space
ex:
======
<html>
  <body>
    <h1>
	&lt;Hello &gt; <Br>
	&amp;<br >
	&copy;
	&trade;
	&reg;<br >
	&euro; &nbsp;  &nbsp;&nbsp;&nbsp;&nbsp;&nbsp;&nbsp;&yen;
  <body>
</html>

CSS 
==========================================
. CSS stands for Cascading Style Sheets.
. CSS defines the structure of html document.
. CSS is to seperate presentation logic(Attributes) with application logic(html tags).
. syntax:
 ----------
    selector1,selector2,...
    {
      name:value; (properties)
      name:value;
      name:value;
    }
ex:
=====
    h1,p
    {
      font-size:100px;
      text-align:center;
      ....
    }
Note:
===========
. selectors are nothing but html tags
. properties are nothinb but attributes
. all properties must be enclosed between left and right braces
. every property contains name and value
. two properties are seperated with semicolan(;)

. Style sheets are of three types.
  1) Internal style sheets
  2) External style sheets
  3) Embedded style sheets

1) Internal style sheets
===========================================================
. We include html code and css code in a single document
. file extension should be .html
. suitable for one web page
. syntax:
==========
  <html>
     <style>
     </style>
     <body>
     </body>
  </html>
.ex:
======
<html>
  <head>
    <style>
	  body
	  {
	    background-color:red;
		background-image:url("image1.jpg");
	  }
	  h1
	  {
	    font-size:100px;
		text-align:center;
		color:blue;
		background-color:black;
		background-image:url("image2.jpg");
	  }
	  p
	  {
	    color:red;font-size:60px;background-color:white;
	  }
	  h2
	  {
	   color:magenta;text-align:center;font-size:100px;
	  }
	</style>
  </head>
  <body>
     <h1>Welcome to CSS</h1>
	 <p>Paragraph tag</p>	
	 <h2>Example for Internal Style Sheets</h2>
  <body>
</html>

External Style sheets
=====================================================
. we define seperate .html and .css files
. suitable for one web application(i.e. collection of web pages)
. External style sheets are much reusable than internal style sheets.

. style.css
=====================
	 body
	  {
	    background-color:red;
		background-image:url("image1.jpg");
	  }
	  h1
	  {
	    font-size:100px;
		text-align:center;
		color:blue;
		background-color:black;
		background-image:url("image2.jpg");
	  }
	  p
	  {
	    color:red;font-size:60px;background-color:white;
	  }
	  h2
	  {
	   color:magenta;text-align:center;font-size:100px;
	  }

index.html
====================
<html>
  <head>
    <link rel="StyleSheet" href="style.css" />
  </head>
  <body>
     <h1>Welcome to CSS</h1>
	 <p>Paragraph tag</p>	
	 <h2>Example for Internal Style Sheets</h2>
  <body>
</html>
