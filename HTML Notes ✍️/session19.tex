String object
========================================================
. String is a collection of characters(i.e alphabets,digits,spaces and special symbols) 
. String can be declared in two ways
  1) using single quotes ex: a='abcde'
  2) using double quotes ex: a="abcde"
. ex:
========
<html>
 <body>
    <font size=7>
     <script>      
	   var a='abcde';
	   document.writeln(a);
	   var b="abcde";
	   document.writeln(b);

	   var c="abcd 1234 #$%^";
	   document.writeln(c);

	   var d=new String("abcde");
	   document.writeln(d);
	</script>
   </font>
  <body>
</html>

String handling functions
===============================================================
1)length: returns number of characters in a given string
-----
ex:
<html>
 <body>
    <font size=7>
     <script>      
	   var a="abcdef";
	   document.writeln(a.length);
	</script>
   </font>
  <body>
</html>

2) toUpperCase(): converts string to upper case
3) toLowerCase(): converts string to lower case
=======================================================
<html>
 <body>
    <font size=7>
     <script>      
	   var a="abcdef";
	   document.writeln(a.toUpperCase());
	   document.writeln(a.toLowerCase());
	</script>
   </font>
  <body>
</html>

4) public char charAt(int index): returns character at that index
5) public int codePointAt(int index): returns ASCII code of that character
==============================================================================
<html>
 <body>
    <font size=7>
     <script>      
	   var a="abcdef";
	   document.writeln(a.charAt(3));
	    document.writeln(a.codePointAt(3));
	</script>
   </font>
  <body>
</html>

6) slice(startindex,endindex)
=======================================================================
. slicing means extracting substring from main string
. returns sub string from startindex to endindex-1
. ex:
========
<html>
 <body>
    <font size=7>
     <script>      
	   var a="abcdefghij";
	   document.writeln(a.slice(3,7));
	   document.writeln(a.slice(4,9));
	</script>
   </font>
  <body>
</html>

7)substring(startindex,endindex):returns substring from startindex to endindex-1
===============================================================================
<html>
 <body>
    <font size=7>
     <script>      
	   var a="abcdefghij";
	   document.writeln(a.substring(3,7));
	   document.writeln(a.substring(4,9));
	</script>
   </font>
  <body>
</html>

8) replace(oldstring,newstring) : replaces old string with new string
9) replaceAll(oldstring,newstring)
==========================================================================
<html>
 <body>
    <font size=7>
     <script>      
	   var a="abcdabcdabcdabcde";
	   document.writeln(a.replace("cd","xyz"),"<br>");
	   document.writeln(a.replaceAll("cd","xyz"));
	</script>
   </font>
  <body>
</html>

10)concat(string): Concatenation is used to append one string to another string
==============================================================================
<html>
 <body>
    <font size=7>
     <script>      
	   var a="Hello";
	   
	   var b=a.concat(" How are you");
	   document.writeln(b,"<br>");

	   var c=a.concat(" Sharadha"," welcome to JS strings");
	   document.writeln(c);

	   var d="abcde";
	   var e="xyz";
	   //var f=d.concat(e);
	   var f = d + e;
	   document.writeln(f);
	</script>
   </font>
  <body>
</html>

11) trim():  removes unnecessary spaces at both ends
12) trimStart(): removes unnecessary spaces at the begining
13) trimEnd(): removes unnecessary spaces at the end
=======================================================================
<html>
 <body>
    <font size=7>
     <script>      
	  var a="hello";
	  var b="   hello    ";
	  document.writeln(a == b,"<br>");
	  document.writeln(a == b.trim() ,"<br>" );

	  var c="   hello";
	  document.writeln(a == c.trimStart())

	  var d="hello    ";
	  document.writeln(a == d.trimEnd())
	</script>
   </font>
  <body>
</html>

14) split(string): divides a string into multiple strings
=======================================================================
<html>
 <body>
    <font size=7>
     <script>      
	  a="red,green,blue,orange";
	  document.writeln(a,"<br>");
	  
	  b=a.split(",");
	  for (i=0;i<b.length;i++)
	  {
		  document.writeln(b[i],"<Br>");
	  }
	  b=a.split(",",2);
	  document.writeln(b);
	</script>
   </font>
  <body>
</html>

Date object
====================================================
. It is used to display system date and time
. syntax: var a=new Date()
. ex:
=======
<html>
 <body>
    <font size=7>
     <script>      
	  var a=new Date();
	  document.writeln(a,"<br>");
	  document.writeln(a.getDate(),"<br>");
	  document.writeln(a.getMonth()+1,"<br>");
	  document.writeln(a.getYear()+1900,"<br>");
	  document.writeln(a.getHours(),"<br>");
	  document.writeln(a.getMinutes(),"<br>");
	  document.writeln(a.getSeconds(),"<br>");
	  document.writeln(a.getTime(),"<br>");
	</script>
   </font>
  <body>
</html>

.getTime(): returns EPOCH/UNIZ standard value. 
EPOCH value: number of milli seconds from 1970 January 1st to till date
. ex-2:
=========
<html>
 <body>
    <font size=7>
     <script>      
	 var a=new Date();
	 var start=a.getTime();
	 for(i=1;i<=100000;i++)
	   document.writeln(i);
	 var b=new Date();
	 var end=b.getTime();
	 document.writeln("Time Take=",end-start);
	</script>
   </font>
  <body>
</html>

Math object
======================
document object
======================


 