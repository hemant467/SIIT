Layers
=============================================
. To work with layers we have to use <div> or <span>
. ex-1:
============
<html>
  <head>
    <style>
	</style>
  </head>
  <body>
    <div style="background-color:red;"><!-- layer1 -->
	  <h1>Welcome to CSS Layers</h1>
	  <img src="image2.jpg" width=300 height=300 />
	</div>
	<div style="background-color:magenta;"><!-- layer2 -->
	   <h2>Layer2</h2>
	   <a href="http://www.google.com">Google</a>
	</div>
  <body>
</html>

Positioning
================================
1) Abolute positioning
2) Relative positioning
3) Overlapping

Absoulte positioning
==============================
attributes
============
position: absolute
top:
bottom: 
left:
right:
width:
height:

ex:
======
<html>
  <head>
    <style>
	  #div1
	  {
	    position:absolute;
		top:100px;
		left:100px;
		width:300px;height:300px;
		background-color:green;
	  }
	  #div2
	  {
	    position:absolute;
		top:100px;
		left:500px;
		width:300px;
		height:300px;
		background-color:cyan;
	  }
	</style>
  </head>
  <body>
       <div id="div1"><!-- layer1 -->
	  <h1>Welcome to CSS Layers</h1>
	  <img src="image2.jpg" width=200 height=200 />
	</div>
	<div id="div2"><!-- layer2 -->
	   <h2>Layer2</h2>
	   <a href="http://www.google.com">Google</a>
	</div>
  <body>
</html>

ex-2:
===================================================
<html>
  <head>
    <style>
	  #header
	  {
	    position:absolute;top:0px;left:0px;width:100%;height:100px;
		background-image:url("image2.jpg");
	  }
	  h1.header
	  {
	    text-align:center; font-size:50px;color:#330033;
	  }
	  #links1
	  {
	    position:absolute;top:110px;left:0px;width:100%;height:80px;
		background-image:url("image3.jpg");text-align:center;
	  }
	  a.links1
	  {
	    font-size:40px;text-decoration:none;margin:20px;
	  }
	  #links2
	  {
	   position:absolute;top:200px;width:200px;height:420px;
	   background-color:pink;
	  }
	  #welcome
	  {
	    position:absolute;top:200px;left:210px;width:800px;height:420px;
		background-color:blue;
	  }
	  #footer
	  {
	   position:absolute;bottom:0px;left:0px;width:100%;
	   font-size:25px;color:red;text-align:center;background-color:black;
	  }
	</style>
  </head>
  <body>
    <div id="header">
	  <h1 class="header">Welcome to SIIT </h1>
	</div>
	<div id="links1">
	   <a href="#" class="links1">Home</a>
	   <a href="contact.html" class="links1">ContactUs</a>
	   <a href="about.html" class="links1">AboutUs</a>
	   <a href="login.html" class="links1">Login</a>
	</div>
	<div id="links2">
	   <a href="#" class="links1">Home</a><br>
	   <a href="contact.html" class="links1">ContactUs</a><br>
	   <a href="about.html" class="links1">AboutUs</a><br>
	   <a href="login.html" class="links1">Login</a>
	</div>
	<div id="welcome">
	  <img src="image1.jpg" width=200 height=200/><br>
	  <h1>Welcome to My Web Page</h1>
	</div>
	<div id="footer">
	   CopyRight@2023, All Rights Reserved
	</div>
  <body>
</html>

2)Relative positioning
================================================
<html>
  <head>
    <style>
	   #header
	   {
	    position:absolute;top:100px;left:100px;
	   }
	   #inner
	   {
	     position:relative;top:-50px;left:50px;
	   }
	</style>
  </head>
  <body>
    <div id="header">
	   <img src="image1.jpg" width=100 height=100 />
	   <div id="inner">
	     <img src="image2.jpg" width=100 height=100 />
	   </div>
	</div>
  <body>
</html>

Overlapping
=============================================
<html>
  <head>
    <style>
	   h1 {font-size:30px;color:red;visibility:visible;}
	   img {position:absolute;top:0px;left:0px;z-index:-1;visibility:visible;}
	</style>
  </head>
  <body>
  <h1>Last example for style sheets</h1>
    <img src="image1.jpg" width=200 height=200 />
	
  <body>
</html>

