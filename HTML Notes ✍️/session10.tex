font properties
==================================
font-size:piexels/em format (1em = 16px (standard font size))
font-weight: bold/bolder
font-style : italic/arial.....
ex:
=====
<html>
  <head>
    <style>
	  h1
	  {
	    font-size:5em;
		font-weight:bolder;
		font-style:italic;
	  }
	</style>
  </head>
  <body>
    <h1>Red color</h1>
  <body>
</html>

text properties
============================================
color: text color
text-decoration: none/underline/overline/line-through
text-transform : uppercase/lowercase
letter-spacing : gap between two letters
word-spacing   : gap between two words
indentation    : left margin
text-shadow    : color x-pos y-pos

ex:
======
<html>
  <head>
    <style>
	  h1
	  {
	    font-size:5em;
		color:red;
		text-align:left;
		text-decoration:underline;
		letter-spacing:5px;
		word-spacing:10px;
		text-indent:200px;
		text-shadow:green 10px 10px
	  }
	</style>
  </head>
  <body>
    <h1>Red color</h1>
  <body>
</html>

margin properties
===============================
margin: px
margin-left:px
margin-right:px
margin-top:
margin-bottom:

ex:
=====
<html>
  <head>
    <style>
	  a 
	  {
		font-size:30px;
		text-decoration:none;	
		margin:20px;
	  }
	</style>
  </head>
  <body>
	<a href="http://www.siit.com">SIIT</a>
	<a href="http://www.google.com">Google</a>
  <body>
</html>


Border properties
========================
border: size color style(solid/dotted/dashed/outset/double)
border-left:
border-right:
border-bottom:
border-top:

ex:
=====
<html>
  <head>
    <style>
	  img 
	  {
		border:15px red double;
		border-bottom:20px green solid;
		border-right:10px blue dashed;
	  }
	</style>
  </head>
  <body>
    <img src="image2.jpg" width=200 height=200 />
  <body>
</html>


Style ID's and Style classes
==============================================
. Style Id's and style classes are used to create user defined selectors.
. Style ID starts with '#' symbol
. Style class starts with '.' symbol
. we can assign more than one value to style class
. ex:
======
<html>
  <head>
    <style>
	  #solid 
	  {
	    border:15px red solid;
	  }
	  .dotted
	  {
	   border:15px green dotted;
	  }
	  .center
	  {
	    text-align:center;
	  }
	</style>
  </head>
  <body>
	<h1 id="solid">Hello</h1>
	<h1 class="dotted">Hi</h1>
	<h1 class="dotted center ">How are you</h1>
  <body>
</html>


ex-2:
===============
<html>
  <head>
    <style>
	  .underline
	  {
	    text-decoration:underline;
	  }
	  h1.red
	  {
	    color:red;
	  }
	  h2#blue
	  {
	   color:blue;
	   font-size:50px;
	  }
	</style>
  </head>
  <body>
	<h1 class="underline red">Heading1</h1>
	<h2 class="underline" id="blue">Heading2</h2>
  <body>
</html>

Layers
===============