window object
=======================
 Methods:
 ----------
	1) alert()
	2) confirm()
	3) prompt()
	4) open()
	5) close()
	6) moveTo(x-pos,y-pos)
ex-1:
=========
<html>
    <head>
	   <script>
	     function disp()
		 {
			window.alert("Welcome to Window object");
			window.confirm("Do you want to continue");
			name = window.prompt("Enter name: ");
			document.writeln(name);
		 }
	   </script>
	</head>
	<body onLoad="disp()">
	<body>
</html>

open()
=============================================
. It opens a new window(pop ups)
. syntax: open("file name","logical name","properties")
. ex:
======
<html>
    <head>
	   <script>
	     function disp()
		 {
			a=window.open("1.html","one",			"width=300,height=300,status=1,resizable=1,menubar=1");	
			a.moveTo(200,300);
			a.close()
		 }
	   </script>
	</head>
	<body onLoad="disp()">
	   <h1>Welcome to Window Pop ups</h1>
	<body>
</html>

location object
======================================
. Used to forward request from one page to another page
. It is a sub object(child object) of window object
.ex:
=======
<html>
    <head>
	   <script>
	    function google()
		{
			a = confirm("Do you to continue");
			if(a == true)
			  window.location = "http://www.google.com";
		}
		function facebook()
		{
			a = confirm("Do you to continue");
			if(a == true)
			  window.location = "http://www.facebook.com";
		}
	   </script>
	</head>
	<body>
	   <form>
	      <input type="button" value="Google" onclick="google() "/>
		  <input type="button" value="Facebook" onclick="facebook()"/>
	   </form>
	<body>
</html>


Q) In html, how many ways are there to forward request from one page to another
===============================================================================
  1) href attribute of hyperlinks 
     <a href="http://goog.com">Google</a>
  2) action attribute of form tag
     <form action="http://google.com">
  3) location object of java script

navigator
=================================
. It gives information about a perticular browser and operating system
. ex:
======
