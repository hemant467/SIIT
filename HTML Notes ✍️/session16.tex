Jump Control statements
=====================================
1) break
2) continue
3) return

break
=================================
. Whenever break statement is executed control comes out of the loop.
. We should use this statement only in loop control statements.
. ex:
======
<html>
  <head>
   <title>WElcome to JS</title>
  </head>
  <body>
    <font size=7>
     <script>
	   for(i=1;i<=10;i++)
	   {
		   if(i==5)
		      break;
		   document.writeln(i);
	   }
	   document.writeln("After for loop");
	</script>
   </font>
  <body>
</html>

continue
==============================
. Whenever continue statement is executed control goes back to loop
. We should use this statement only in loop control statements.
. ex:
======
<html>
  <head>
   <title>WElcome to JS</title>
  </head>
  <body>
    <font size=7>
     <script>
	   for(i=1;i<=10;i++)
	   {
		   if(i==5)
		      continue;
		   document.writeln(i);
	   }
		document.writeln("<Br>");
	   for(i=1;i<=10;i++)
	   {
		   //&& and (both conditions are true then result is true)
		   //|| or  (if any one condition is true then result is true)
		   if(i==2 || i==6 || i==8)
		      continue;
		   document.writeln(i);
	   }
	</script>
   </font>
  <body>
</html>

ex-2: WAP to display even numbers from 1 to 100
============================================================
<html>
  <head>
   <title>WElcome to JS</title>
  </head>
  <body>
    <font size=7>
     <script>
	   for(var i=1;i<=100;i++)
	   {
		   if(i%2 != 0)
		       continue;
		   document.writeln(i);
	   }
	</script>
   </font>
  <body>
</html>

Functions
=============================================================================
. Function is a block of a code that performs specific task.
. Functions are also called "Methods" or "Procedures".
. In java script functions are declared with a keyword called "function".
. syntax:
----------
   function functionname(arguments/parameters)
   {
     ....body/definition of 
     ....function
   }

ex:
======
<html>
  <head>
   <title>WElcome to JS</title>
  </head>
  <body>
    <font size=7>
     <script>
	   function welcome() //called function
	   {
		   document.writeln("Hello, Welcome to JS functions");
	   }
	   welcome() //calling function
	</script>
   </font>
  <body>
</html>

1) We may have any number of functions
2) one function can call another function
3) one function can call itself is called "Recursion".
4) we may not define as a sequence called in our program.

. ex:
=======
<html>
  <head>
   <title>WElcome to JS</title>
  </head>
  <body>
    <font size=7>
     <script>
	 function arithmetic()
	 {
		  add();
		  sub();
	 }
	 function add()
	 {
		 a=parseInt(prompt("Enter a: "))
		 b=parseInt(prompt("Enter b: "))
		 document.writeln(a+b);
	 }
	function sub()
	 {
		 a=parseInt(prompt("Enter a: "))
		 b=parseInt(prompt("Enter b: "))
		 document.writeln(a-b);
	 }
	 arithmetic()
	</script>
   </font>
  <body>
</html>

. area of program using functions
. average of 5 numbers

. Functions are divided into two types
  1) System defined functions
  2) User defined functions

. If the body of the function is given by the system is called system defined function
  ex: writeln(), prompt(), alert(), confirm(), parseInt()

. If the body of the function is given by the user is called User defined function
  ex:  arithmetic(), add(), sub(), welcome()


 
