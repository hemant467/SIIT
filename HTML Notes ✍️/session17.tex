factorial program
--------------------------------------------
<script>
  var n=parseInt(prompt("enter n: "))
  var fact=1;
  for(var i=1;i<=n;i++)
  {
    fact = fact * i;
  }
  document.writeln(fact);
</script>

.Area of rectangle program using functions
===============================================================
<html>
  <head>
   <title>WElcome to JS</title>
  </head>
  <body>
    <font size=7>
     <script>
		function area()
		{
			var l=parseInt(prompt("Enter l: "));
			var b=parseInt(prompt("Enter b: "));
			var a=l*b;
			document.writeln(a);
		}
		area();
		area();
		area();
	</script>
   </font>
  <body>
</html>

.Average of 5 subjects
==================================
avg = (a+b+c+d+e)/5
ex:
=================================

<html>
  <head>
   <title>WElcome to JS</title>
  </head>
  <body>
    <font size=7>
     <script>
		function average()
		{
			var a=parseInt(prompt("Enter a"));
			var b= parseInt(prompt("Enter b"));
			var c=parseInt(prompt("Enter c"));
			var d=parseInt(prompt("Enter d"));
			var e=parseInt(prompt("Enter e"));
			var avg = (a+b+c+d+e)/5;
			document.writeln(avg);
		}
		average();
	</script>
   </font>
  <body>
</html>
. WAP to find simple interest
=================================
si=p*t*r/100

Argumentst(Parameters)
=======================================================================
.Values passing from calling function to called function are called arguments.
.We can send any number of arguments but number of sending arguments must be equal to number of receiving arguments.
.Sending arguments are said to be actual arguments/parameters
.Receiving arguments are said to be formal arguments
.actual and formal argument names may or may not be same
.ex-1:
========
<html>
  <head>
   <title>WElcome to JS</title>
  </head>
  <body>
    <font size=7>
     <script>       //formal arguments
		function add(a,b)
		{
			document.writeln(a+b);
		}
		function sub(a,b)
		{
			document.writeln(a-b);
		}
		a=100
		b=200
		add(a,b) //actual arguments
		sub(a,b)
	</script>
   </font>
  <body>
</html>

ex-2: WAP to find area of rectangle(a=l*b)
=============================================
<html>
  <head>
   <title>WElcome to JS</title>
  </head>
  <body>
    <font size=7>
     <script>       //formal arguments
		function area(l,b)
		{
		  var a=l*b;
		  document.writeln(a);
		}	
		l=parseInt(prompt("Enter l: "))
		b=parseInt(prompt("Enter b: "))
		area(l,b)
	</script>
   </font>
  <body>
</html>

average 5 numbers
simple interest using arguments

Return type
======================================
. whenever return statement is executed contol returns to calling function
. we can return only one value
 ex:
    return a;
    return a+b;
    return a,b;Xerror
ex:
====
<html>
  <head>
   <title>WElcome to JS</title>
  </head>
  <body>
    <font size=7>
     <script>       
		function add(a,b)
		{
			var c=a+b;
			return c;
		}
		function sub(a,b)
		{
			var c=a-b;
			return c;
		}

		a=10
		b=20
		c=add(a,b)
		document.writeln(c);

		c=sub(a,b)
		document.writeln(c);
	</script>
   </font>
  <body>
</html>

. Average of 5 numbers
=====================================
<html>
  <head>
   <title>WElcome to JS</title>
  </head>
  <body>
    <font size=7>
     <script>       
		function average(a,b,c,d,e)
		{
			avg = (a+b+c+d+e)/5;
			return avg;
		}
		avg = average(10,20,30,40,50);
		document.writeln(avg);
	</script>
   </font>
  <body>
</html>

(or)
<html>
  <head>
   <title>WElcome to JS</title>
  </head>
  <body>
    <font size=7>
     <script>       
		function average(a,b,c,d,e)
		{
			avg = (a+b+c+d+e)/5;
			return avg;
		}
		var a=parseInt(prompt("Enter a"));
		var b= parseInt(prompt("Enter b"));
		var c=parseInt(prompt("Enter c"));
		var d=parseInt(prompt("Enter d"));
		var e=parseInt(prompt("Enter e"));
		avg = average(a,b,c,d,e);
		document.writeln(avg);
	</script>
   </font>
  <body>
</html>

Objects in java script
==============================
1) Array object
2) String object
3) Date
4) Math
5) document
6) window
7) location
8) navigator
9) history
10) screen